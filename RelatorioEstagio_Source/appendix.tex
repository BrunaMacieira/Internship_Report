\chapter{Additional Information} \label{add_info}

Here, it's presented additional information about permissionless and permissioned blockchain and consensus mechanisms.

\begin{table}[h]
\centering
\begin{tabular}{|l|p{5.5cm}|p{5.5cm}|}
\hline
 & \textbf{Permissionless} & \textbf{Permissioned} \\
\hline
\textbf{Overview} & Open Network allowing anyone to interact, consensus validation, and fully decentralized & Closed Network with limited decentralization and designated parties for participation in Consensus Validation \\
\hline
\textbf{Also Known As} & Public, Trustless & Private, Permissioned Sandbox \\
\hline
\textbf{Key Attributes} & 
\begin{itemize}
  \item Full transparency
  \item Development in Open Source
  \item Mostly Anonymous
  \item Privacy dependent on technological limitations
  \item No Control Authority
  \item Involves digital assets
\end{itemize} 
& 
\begin{itemize}
  \item Controlled Transparency
  \item Development by Private entities
  \item Not anonymous
  \item Privacy based on governance decision
  \item No Single authority
  \item May or may not involve digital assets
\end{itemize} \\
\hline
\textbf{Benefits} & 
\begin{itemize}
  \item Broader decentralization
  \item Highly transparent
  \item Censorship resistant
  \item Security resilience
\end{itemize} 
& 
\begin{itemize}
  \item Incremental Decentralization
  \item Strong privacy
  \item Customizable
  \item Faster and Scalable
\end{itemize} \\
\hline
\textbf{Drawbacks} & 
\begin{itemize}
  \item Less energy efficient
  \item Slow and difficult to scale
  \item Less user privacy
\end{itemize} 
& 
\begin{itemize}
  \item Limited decentralization
  \item Override risk
  \item Less transparent
\end{itemize} \\
\hline
\textbf{Market Traction} & P2P, B2C, Government to Citizens & B2B, B2C, Government to Organizations \\
\hline
\end{tabular}
\captionsetup{justification=centering}
\caption{Permissionless and Permissioned Blockchains Comparison, source: \href{https://www.blockchain-council.org/blockchain/permissioned-and-permissionless-blockchains-a-comprehensive-guide/}{Blockchain Council}}\footnotemark{}
\label{tab:blockchain-comparison}
\end{table}


\begin{table}[h]
\centering
\begin{tabular}{|p{2.5cm}|p{5cm}|p{5cm}|}
\hline
\textbf{Blockchain Type} & \textbf{Pros} & \textbf{Cons} \\
\hline
\multicolumn{1}{|c|}{\textbf{Permissioned}} & 
\begin{itemize}
  \item Risk-free participation without reliance on centralized models.
  \item Strong privacy protection and customizable configurations.
  \item Improved performance and scalability due to limited participants.
\end{itemize} 
& 
\begin{itemize}
  \item Higher chances of collusion and data corruption.
  \item Difficulties in achieving consensus with potentially changeable rules.
  \item Limited transparency for external oversight.
\end{itemize} \\
\hline
\multicolumn{1}{|c|}{\textbf{Permissionless}} & 
\begin{itemize}
  \item Broader decentralization with high transparency and resistance to censorship.
  \item Efficient reconciliation process and resistance to data disruption.
\end{itemize} 
& 
\begin{itemize}
  \item Lower privacy compared to permissioned blockchain.
  \item Potential performance and scalability issues due to increased user base.
  \item Higher energy consumption for network-wide verification.
\end{itemize} \\
\hline
\end{tabular}
\captionsetup{justification=centering}
\caption{Pros and Cons of Permissioless and Permissioned Blockchains, source: \href{https://www.blockchain-council.org/blockchain/permissioned-and-permissionless-blockchains-a-comprehensive-guide/}{Blockchain Council}}\footnotemark[\value{footnote}]
\label{proscons}
\end{table}

\footnotetext[1]{For more information, please consult: https://www.blockchain-council.org/blockchain/permissioned-and-permissionless-blockchains-a-comprehensive-guide/}

\begin{table}[h]
\centering
\begin{tabular}{|p{3cm}|p{3.5cm}|p{3.5cm}|p{3cm}|}
\hline
\textbf{Consensus Mechanism} & \textbf{Advantages} & \textbf{Disadvantages} & \textbf{Protocols Using It} \\
\hline
Proof of Work & 
Decentralized Structure\newline High Levels Of Security\newline Acceptable Levels Of Scalability & High Block Time\newline Energy Inefficiency\newline Hardware Dependency & Bitcoin\newline Dogecoin\newline Litecoin \\
\hline
Proof of Stake & 
Fast Block Creation Time\newline High Throughput\newline Energy Efficiency\newline Scalability (But Less Than PoW)\newline Independence to Special Hardware & Suffers From Centralization\newline Lower Cost Of Misbehaving & Tezos\newline Cardano\newline Ethereum \\
\hline
Delegated Proof of Stake & 
Scalability\newline Energy Efficiency\newline Low-cost Transactions & Semi-centralization\newline Highly Susceptible To 51\% Attack & EOS\newline Ark\newline Tron \\
\hline
Practical Byzantine Fault Tolerance & 
Energy Efficiency\newline High Throughput & Not Scalable\newline Susceptible To Sybil Attacks & Hyperledger Fabric\newline Zilliqa \\
\hline
Proof Of Weight & 
Great Customization and Scalability\newline Quick Transaction Confirmation\newline Energy Efficiency & No Incentive\newline Semi-centralization & Algorand\newline Filecoin\newline Chia \\
\hline
Proof Of Capacity & 
No Special Hardware\newline More Decentralized & Susceptible To Grinding Attack\newline Space Privilege Applies & Burstcoin\newline Permacoin \\
\hline
Proof Of Authority & 
Transactional Speed\newline Tighter Security & Not Decentralized\newline Breaks Anonymity & VeChain\newline Palm Network\newline Xodex \\
\hline
Proof of Importance & 
Sybil Resistant\newline Prevents Hoarding & Favors The Rich\newline Little Incentives & NEM \\
\hline
\end{tabular}
\captionsetup{justification=centering}
\caption{Consensus Mechanisms in Blockchain, source: \href{https://hacken.io/discover/consensus-mechanisms/}{Hacken}}\footnotemark
\label{tab:consensus-mechanisms}
\end{table}

\footnotetext{For more information, please consult: https://hacken.io/discover/consensus-mechanisms/}

