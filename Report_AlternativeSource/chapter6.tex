\chapter{Conclusions and future work} \label{chap:chap6}

In this final chapter, we will analyze the development of the project so far and draw conclusions based on the findings. We will also discuss future work and potential areas for improvement.

\section{Achievement of goals}

Working on this project has been an incredibly rewarding experience, as it has provided an opportunity to gain skills and knowledge that were previously thought to be attainable only through a master's degree. The project has allowed for personal growth and the development of essential technical skills.\newline

Throughout the project, there were moments of self-doubt and the feeling of being inadequate for the task at hand. However, overcoming challenges and successfully resolving issues, such as the bug encountered in section \ref{bug}, provided a sense of accomplishment and confidence in the project's progress.\newline

Additionally, the project has facilitated the improvement of collaboration skills, particularly in working with version control systems like GitHub. The ability to effectively manage and collaborate on code repositories is a valuable skill in the modern software development landscape.\newline

Given that the project deviated from the original sprint planning, it becomes challenging to definitively assess the achievement of the goals. However, considering the work undertaken, it is apparent that significant progress was made towards attaining the goals and the project served as an invaluable catalyst for personal and professional development.\newline

\section{Future work}

\\As Dharma Network progresses, future work will focus on several key areas. The establishment of a Dharma Network Foundation will provide legal support and transition towards a Decentralized Autonomous Organization (DAO). The Dharma DAO will enable token holders to participate in governance and decision-making processes. Furthermore, the expansion of liquidity pools, fee adjustments, and grants for projects aligned with the Dharma Network's vision will be explored in later phases.\\

Moving forward with Dharma Network, there are several areas that warrant further attention and future work. These include:

\begin{enumerate}
    \item \textit{Establishing a Dharma Network Foundation:} Creating a legal entity to support the Dharma Network will provide a solid foundation for governance and facilitate the transition towards a Decentralized Autonomous Organization (DAO). The DAO structure will enable token holders to participate in decision-making processes and shape the future direction of the project.
    \item \textit{Expanding Liquidity Pools and Fee Adjustments:} The project should explore strategies to increase liquidity pools within the Dharma Network. This can be achieved through partnerships, incentivizing liquidity providers, and adjusting transaction fees to maintain a healthy and sustainable ecosystem.
    \item \textit{Grants for Aligned Projects:} Offering grants and support to projects aligned with the Dharma Network's vision will foster innovation and growth within the ecosystem. This will encourage developers and entrepreneurs to build on the platform and contribute to its expansion.
\end{enumerate}

In conclusion, future work should focus on governance, liquidity expansion and fostering a vibrant ecosystem of projects within the Dharma Network.

\section{The Future of DeFi}

The future of decentralized finance holds immense potential for reshaping the global financial landscape and addressing income inequality. By providing accessible, transparent and inclusive financial services, DeFi has the power to empower individuals and communities that have been historically marginalized. However, several challenges and considerations must be addressed to ensure the long-term success and sustainability of the DeFi ecosystem.\newline

Technological barriers, regulatory frameworks and infrastructure limitations remain as hurdles that need to be overcome to achieve widespread adoption and inclusivity. Efforts must be made to bridge the digital divide and provide education and awareness about DeFi to underserved communities. Collaboration between governments, financial institutions and technology innovators is crucial to harness the full potential of decentralized finance in tackling global income inequality.\newline

The governance structures within DeFi projects need careful attention to ensure accountability, transparency and fairness. Clear mechanisms for decision-making and community involvement are vital to protect the interests of participants and maintain the integrity of the network. Additionally, addressing potential risks associated with the gamification of finance and promoting responsible market behavior will contribute to the long-term stability of the DeFi ecosystem \cite{oecd}.\newline

Despite these challenges, the transformative power of decentralized finance should not be underestimated. As we navigate the evolving DeFi landscape, it is essential to prioritize innovation, collaboration, and responsible development. By addressing these challenges and fostering an inclusive and sustainable DeFi ecosystem, we can ensure that the benefits of decentralized finance are shared by all, contributing to a more equitable and inclusive financial future.\newline


\begin{comment}
    However, challenges remain. Technological barriers, regulatory frameworks and infrastructure limitations, such as poor or no access to the Internet, must be addressed to ensure widespread adoption and inclusivity. Moreover, efforts must be made to bridge the digital divide and provide education and awareness about DeFi to underserved communities. Collaboration between governments, financial institutions and technology innovators is crucial to harness the full potential of decentralized finance in tackling global income inequality.

Decentralized finance and cryptocurrencies have the potential to reshape the global financial landscape and address the persistent issue of income inequality. By providing accessible, transparent, and inclusive financial services, DeFi can empower individuals and communities that have been historically marginalized. While challenges exist, the transformative power of decentralized finance should not be underestimated. As we navigate this evolving landscape, it is essential to prioritize innovation, collaboration, and responsible development to ensure that the benefits of DeFi are shared by all.
\end{comment}

Moreover, one of the significant challenges faced by the decentralized finance industry is the carbon emissions associated with the mining processes. As the environmental impact of blockchain networks becomes more apparent, addressing carbon footprints has become a pressing concern. In this regard, the next significant step for Dharma Network could be to explore ways of making the mining process of their token carbon-neutral.\newline

The significance of addressing carbon emissions in the blockchain space becomes evident when compared to the energy consumption of entire countries. For instance, Bitcoin consumes around 130 terawatt-hours (TWh) of energy annually, which is comparable to the energy consumption of countries like Sweden (131 TWh), Norway and Argentina (both around 125 TWh). It is nearly triple the energy consumption of Portugal, which is approximately 48 TWh per year \cite{ns}.\newline

Algorand's energy consumption is known to be relatively low compared to other blockchain networks. According to the CTO of Algorand Foundation, Algorand uses only 80 kW of energy and incorporates carbon offsetting measures into its operations, making it one of the most eco-friendly blockchain networks available \cite{daddy, tel}.\newline

What about the carbon footprint of Dharma Network? Although it's unknown, Dharma Network is built on top of the Algorand blockchain, so it's clear to say that Dharma Network can also be considered a low carbon footprint project. 


\section{Conclusion}

In conclusion, Dharma Network represents an innovative and disruptive solution in the realm of decentralized finance. By combining the power of the Algorand blockchain, robust backend services and a community-driven governance structure, Dharma Network strives to democratize access to financial services and foster a truly decentralized and inclusive ecosystem.\newline

Dharma Network has the potential to empower employees by embracing the ethos that "(...) graduate worker or master's degree holder, we are all workers" (José Teixeira, DST's CEO, in \textit{"É ou Não É"}, RTP). In this paradigm, the traditional barriers that differentiate individuals based on their educational backgrounds are diminished, as the focus shifts to valuing the contributions and skills of every worker. By providing equal opportunities for participation and leveraging the power of blockchain technology, Dharma Network enables individuals to harness their potential and compete on a level playing field. As José Teixeira, DST's CEO, aptly stated, "(...) the freer individuals are, the more competitive they tend to be", which also brings an advantage to the company, since "(...) we use resources because it brings a competitive advantage" (José Teixeira, DST's CEO, in \textit{"É ou Não É"}, RTP). With Dharma Network, employees can leverage their skills, tap into a decentralized financial ecosystem and capitalize on resources to gain a competitive advantage in the global marketplace. \newline

\begin{comment}
    There are certain issues that may occur with such projects:

\begin{enumerate}
    \item \textbf{Lack of Accountability:}\newline

    \textit{Problem:}
    
    The decentralized nature of DeFi projects can lead to a lack of accountability for those launching open protocols. Initial VC (Venture Capital\footnote{Venture Capital is a form of private equity and financing that investors provide to businesses they believe to have long-term growth}) funders and core developers often retain governance tokens, but the size of their holdings may not be transparent to users or regulators \cite{oecd}.

    \textit{Dharma Network's solution:}

    \item \textbf{Community Splits and Forks:}\newline

    \textit{Problem:}

     Contentious decisions within the community can result in network splits and forks. In proof-of-work systems, like Bitcoin and Ethereum, miners with majority control can influence transaction validation and the future direction of the chain. It may be unclear how protocol changes affecting existing contracts are decided by the community \cite{oecd}.

    \textit{Dharma Network's solution:}

    \item \textbf{Gamification of Finance:}\newline

    \textit{Problem:}

     The gamification of DeFi platforms can attract inexperienced investors, potentially leading to irrational trading and reliance on unsound financial advice. These issues relate to consumer protection and may impact market behavior and financial stability. Loss of confidence in DeFi could affect traditional financial systems \cite{oecd}.

    \textit{Dharma Network's solution:}
\end{enumerate}
\end{comment}





